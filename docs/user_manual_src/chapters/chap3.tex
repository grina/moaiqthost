\chapter{Using and configuring Moai Qt Host}
\label{chap:using}

\section{General}
\label{sec:gen}

In general, just give your Lua scripts for the host as command line arguments. 

Moai Qt Host requires Qt libraries in order to work, so you must make sure that they are available.
To do this you can, for example, include the libraries in the same folder as the Moai Qt Host executable, or add the libraries path to environment variables.



\section{Compile Configurations}
\label{sec:compile_config}

Moai Qt Host can be configured for different purposes by qmake configurations.
Different parameters are explained in table~\ref{config_params} on page~\pageref{config_params}.
The ''Dependency libraries''-column refers to the libraries that should be copied to ''libs''-folder (refer to section~\ref{sec:buildhost} step 6) when using corresponding configuration.

Enabling each parameter is done by adding them to the \texttt{CONFIG}-variable in \textit{moai\_root.pro}-project file.
For Windows default configuration, which has the \texttt{luaext} and \texttt{particle\_presets} --parameters enabled, this would look like following: 
\begin{quote}
\texttt{CONFIG += luaext particle\_presets}
\end{quote}

\pagebreak % To put OSX & Linux configs on one page.
Default configuration for OS X and Linux are:
\begin{quote}
OS X:  \texttt{CONFIG += luaext particle\_presets untz}\\
Linux: \texttt{CONFIG += untz}
\end{quote}

To get the required libraries and to ensure compatibility, the Moai SDK should be built with same kind of configuration as the Moai Qt Host. 
So, for example, if you want to use FMOD Ex --sound system, you should build the Moai SDK with FMOD Ex configuration and use the libraries from that build, in addition to setting the \texttt{"fmod\_ex"}-parameter for Moai Qt Host build.

\begin{table}[h!]
\begin{center}
\caption{Configuration parameters for Moai Qt Host}
\label{config_params}
\begin{tabular}{|l|p{0.35\textwidth}|p{0.35\textwidth}|}
\hline
Parameter & Description & Dependency libraries\\ 
\hline
\hline
audiosampler & Enables the MOAIAudioSampler-extension. Not fully functional at the time of writing this document. & ---\\
\hline
debugger & Enable the Moai SDK harness for external debuggers. Using 3rd party debuggers (such as ZeroBrane Studio) with Moai Qt Host has not been tested and might not work. & \texttt{moaiext-debugger}\\
\hline
fmod\_designer & Use FMOD Designer as the sound system (Not tested). & From Moai SDK: \texttt{moaiext-fmod-designer}\newline From FMOD: \texttt{fmodex\_vc}\\
\hline
fmod\_ex & Use FMOD Ex as the sound system (Not tested on Linux). & \textbf{Windows \& Linux}: \newline From Moai SDK: \texttt{moaiext-fmod-ex}\newline From FMOD: \texttt{fmodex\_vc} \newline \textbf{OS X}: \newline From Moai SDK: \texttt{moai-osx-fmod-ex} \newline From FMOD: \texttt{libfmodex}\\
\hline
luaext & Enable the Lua-extensions for Moai SDK. & \textbf{Windows \& Linux}: \texttt{moaiext-luaext} \newline \textbf{OS X}: \texttt{moai-osx-luaext}\\
\hline
particle\_presets & Use custom handler (\texttt{ParticlePresets}-C++-files) for particle systems. & ---\\
\hline
untz & Use UNTZ as the sound system. & \textbf{Windows}: \texttt{moaiext-untz} \newline \textbf{OS X}: \texttt{moai-osx-untz} \newline \textbf{Linux}: \texttt{moaiext-untz} and \texttt{untz}\\
\hline
\end{tabular}
\end{center}
\end{table}
