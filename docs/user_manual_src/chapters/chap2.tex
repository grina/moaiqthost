\chapter{Setting up Moai Qt Host}
\label{chap:setupshort}

To set up Moai Qt Host you have two options. You can
\begin{enumerate}[(a)]
\item Dowload Moai Qt Host binary and use it as is, or
\item Compile Moai SDK and the Moai Qt Host yourself. 
\end{enumerate}

\section{Option A: Download binary}

This is the simplest way to go. Download Moai Qt Host binary. You must have the necessary Qt libraries on your computer (see ~\ref{sec:gen}).
The binaries are located at \textit{''moaiqthost/bin/''} -folder.

\section{Option B: Compile Moai Qt Host yourself}
\label{sec:buildhost}

This section describes how to do the debug build of Moai Qt Host. 
The release build is built similarly. 
Just make sure you build both the Moai SDK and the Moai Qt Host with same settings.

\begin{enumerate}
\item Set up Qt: the build has been tested with the open source version of Qt SDK 1.2.1 (libraries version 4.8.1, Qt Creator version 2.4.1), downloadable from \url{http://qt-project.org/}. On Windows and OS X you must use the 32-bit version of Qt libraries (on OS X you need to build these yourself). The Linux build has only been tested with 64-bit libraries.

\item Download Moai Qt Host source.

\item Download the Moai SDK. The supported version is 1.3 (Build 160). The source can be found from:
	\begin{enumerate}
	\item \textbf{On Windows and OS X}: \url{https://github.com/moai/moai-dev}
	\item \textbf{On Linux}: \url{https://github.com/spacepluk/moai-dev} (the supported version is dated at January 22nd 2013)
	\end{enumerate}
	
\item Place the Moai SDK source to \textit{''moaiqthost/src/moaipackage''} folder.

\item Compile Moai SDK. Note that Moai SDK and the host must have the same compile configuration (for more information, refer to section~\ref{sec:compile_config}).
	\begin{enumerate}
	\item \textbf{On Windows}: build with Microsoft Visual Studio 2010, according to instructions in \textit{''moaiqthost/src/moaipackage/vs2010/README.txt''}.
	\item \textbf{On OS X}: build with Xcode or Clang, according to instructions in\\ \textit{'''moaiqthost/src/moaipackage/xcode/README''}.
	\item \textbf{On Linux}: build according to instructions in\\ \textit{''moaiqthost/src/moaipackage/README.md''}.
	\end{enumerate}
	
\item Copy the compiled Moai SDK libraries to \textit{''moaiqthost/src/moaiqthost/libs''} -folder. You can find the libraries from:
	\begin{enumerate}
	\item \textbf{On Windows}: \textit{''moaiqthost/src/moaipackage/vs2010/bin/Win32/Debug'}'
	\item \textbf{On OS X}: \textit{''/Users/<username>/Library/Developer/Xcode/DerivedData\\/libmoai-<random text>/Build/Products/Debug''}
	\item \textbf{On Linux}: inside the subfolders of  \textit{''moaiqthost/src/moaipackage/build/src/''} and \textit{''moaiqthost/src/moaipackage/build/3rdparty/''} -- folders.
	\end{enumerate}
	
\item Open the Moai Qt Host --project from \textit{''moaiqthost/src/moaiqthost/moaiqthost.pro''}.

\item Configure the Moai Qt Host build if needed (refer to chapter~\ref{chap:using}).

\item In Qt Creator under ''Projects'', make sure you are using the same compiler that you built Moai SDK with:
	\begin{enumerate}
	\item \textbf{On Windows}: MSVC2010
	\item \textbf{On OS X}: GCC 4.2
	\item \textbf{On Linux}: GCC 4.6.3
	\end{enumerate}
	
\item Build. The resulting executable file is named ''moaiqt'', and will appear in:
	\begin{enumerate}
	\item \textbf{On Windows and OS X}: \textit{''moaiqthost/src/moaiqthost/Debug''} -- folder
	\item \textbf{On Linux}: \textit{''moaiqthost/src/moaiqthost/''} -- folder
	\end{enumerate}
\end{enumerate}

\section{About audio}
\label{sec:aboutaudio}

Moai offers two options for implementing audio. 
Zipline recommends using the open source UNTZ audio library. 

The second option is to use FMOD audio library, which is not free for commercial development.
Enabling either option in Moai Qt Host is described in chapter~\ref{chap:using}. FMOD Ex is the current FMOD version.

Note: Fmod requires runtime libraries. These are fmodex.dll on Windows  and libfmodex.dylib on OS X.
